\documentclass{article}

\usepackage{fancyhdr}
\usepackage{wrapfig}
\usepackage{extramarks}
\usepackage{multicol}
\usepackage{amsmath}
\usepackage{amsthm}
\usepackage{amsfonts}
\usepackage{tikz}
\usepackage{forest,adjustbox}
\usepackage[plain]{algorithm}
\usepackage{algpseudocode}
\usepackage{enumerate}
\usepackage[shortlabels]{enumitem}                     
          \setlist[enumerate, 1]{1\textsuperscript{o}}



\usepackage{listings}
\usepackage{xcolor}
\lstset { %
    language=python,
    frame=single,
    showstringspaces=false,
    backgroundcolor=\color{black!5}, % set backgroundcolor
    basicstyle=\footnotesize,% basic font setting
}

%\usetikzlibrary{automata,positioning}
\usetikzlibrary{positioning,shapes,shadows,arrows,automata} % % Basic Document Settings %

\topmargin=-0.45in
\evensidemargin=0in
\oddsidemargin=0in
\textwidth=6.5in
\textheight=9.0in
\headsep=0.25in

\linespread{1.1}

\pagestyle{fancy}
\lhead{\hmwkClass}
\chead{ (\hmwkClassInstructor\ \hmwkClassTime) }
\rhead{\shortName \hspace{0.4cm} \hmwkTitle}
\lfoot{\lastxmark}
\cfoot{\thepage}

\renewcommand\headrulewidth{0.4pt}
\renewcommand\footrulewidth{0.4pt}

\newcommand\curl[1]{\{#1\}}

\setlength\parindent{0pt}

%
% Create Problem Sections
%

\newcommand{\enterProblemHeader}[1]{
  \nobreak\extramarks{}{Problem \arabic{#1} continued on next page\ldots}\nobreak{}
  \nobreak\extramarks{Problem \arabic{#1} (continued)}{Problem \arabic{#1} continued on next page\ldots}\nobreak{}
}

\newcommand{\exitProblemHeader}[1]{
  \nobreak\extramarks{Problem \arabic{#1} (continued)}{Problem \arabic{#1} continued on next page\ldots}\nobreak{}
  \stepcounter{#1}
  \nobreak\extramarks{Problem \arabic{#1}}{}\nobreak{}
}

\setcounter{secnumdepth}{0}
\newcounter{partCounter}
\newcounter{homeworkProblemCounter}
\setcounter{homeworkProblemCounter}{1}
\nobreak\extramarks{Problem \arabic{homeworkProblemCounter}}{}\nobreak{}

%
% Homework Problem Environment
%
% This environment takes an optional argument. When given, it will adjust the
% problem counter. This is useful for when the problems given for your
% assignment aren't sequential. See the last 3 problems of this template for an
% example.
%
\newenvironment{homeworkProblem}[1][-1]{
  \ifnum#1>0
    \setcounter{homeworkProblemCounter}{#1}
  \fi
  \section{Problem \arabic{homeworkProblemCounter}}
  \setcounter{partCounter}{1}
  \enterProblemHeader{homeworkProblemCounter}
}{
  \exitProblemHeader{homeworkProblemCounter}
}

%
% Homework Details
%  - Title
%  - Due date
%  - Class
%  - Section/Time
%  - Instructor
%  - Author
%

\newcommand{\hmwkTitle}{Homework\ \#6}
\newcommand{\hmwkDueDate}{March 16 2016}
\newcommand{\hmwkClass}{CS581 Theory of Computation}
\newcommand{\hmwkClassTime}{Winter 2016}
\newcommand{\hmwkClassInstructor}{Harry H. Porter}
\newcommand{\hmwkAuthorName}{Konstantin Macarenco}
\newcommand{\shortName}{Konstantin M.}

%
% Title Page
%

\title{
  \vspace{2in}
  \textmd{\textbf{\hmwkClass:\ \hmwkTitle}}\\
  \normalsize\vspace{0.1in}\small{Due\ on\ \hmwkDueDate\ at 12:30pm}\\
  \vspace{0.1in}\large{\textit{\hmwkClassInstructor\ \hmwkClassTime}}
  \vspace{3in}
}

\author{\textbf{\hmwkAuthorName}}
\date{}

\renewcommand{\part}[1]{\textbf{\large Part \Alph{partCounter}}\stepcounter{partCounter}\\}
\newcommand{\answ}[1]{\hspace{1cm}\textbf{Answ:} #1}

\newcommand{\domino}[2]{\left [ \cfrac{#1}{#2} \right ]}
\newcommand{\tile}[2]{\cfrac{#1}{#2}}
\newcommand{\problem}[1]{\large{\textbf{Problem #1}}\\}
\newcommand{\imp}[1]{\textbf{#1}}
\newcommand{\nth}{$ Th(N,+,\times) $}

%
% Various Helper Commands
%

% Useful for algorithms
\newcommand{\alg}[1]{\textsc{\bfseries \footnotesize #1}}

% For derivatives
\newcommand{\deriv}[1]{\frac{\mathrm{d}}{\mathrm{d}x} (#1)}

% For partial derivatives
\newcommand{\pderiv}[2]{\frac{\partial}{\partial #1} (#2)}

% Integral dx
\newcommand{\dx}{\mathrm{d}x}

% Alias for the Solution section header
\newcommand{\solution}{\textbf{\large Solution}}

% Probability commands: Expectation, Variance, Covariance, Bias
\newcommand{\E}{\mathrm{E}}
\newcommand{\Var}{\mathrm{Var}}
\newcommand{\Cov}{\mathrm{Cov}}
\newcommand{\Bias}{\mathrm{Bias}}

\newcommand\Vtextvisiblespace[1][.3em]{%
  \mbox{\kern.06em\vrule height.3ex}%
  \vbox{\hrule width#1}%
  \hbox{\vrule height.3ex}}

\begin{document}

\maketitle

\pagebreak

\textbf{Chapter 6.2 Decidability of logical theories}

\begin{enumerate}[1., leftmargin = 0.6cm]
\itemsep0em
\item Well formed formula\\
    A formula is a well-formed string over this 
    \begin{enumerate}[1., leftmargin = 0.6cm]
    \itemsep0em
        \item is an atomic formula
        \item has the form $\phi \wedge \phi$ or $\phi \vee \phi$ or $\neg \phi$ and $\phi$ are 
            smaller formulas, or
        \item has the form $\exists x_i [\phi_1] or \forall x_i[\phi_1]$ where $\phi_1$ is a
            smaller formula.
    \end{enumerate}
    \item prenex normal form - all quantifiers apper in the front of the formula.
    \item a variable isn't bound withing the scope of a quantifier is called a free variable
    \item formula with no free variables is called a sentence of statement.
    \item \textbf{universe} with  assignment of relations to relation symbols is called a \textbf{model}

    \item if $M$ is a model we let the  \imp{theory\ of \ M}, written $Th(M)$, be the collection of
        true sentence in the language of that model

    \item \imp{A DECIDABLE THEORY}
    \item Tuatology always true in \underline{any} model
    \item Axioms a given set of statements assumed to be true without proof.
    \teim Rules of inference/deduction

        
    \item \imp{Kurt Godel} showed that no algorithm can decide in general whether statements in number theory are true or false.
        \item \imp{Church showed that } $Th(N,+,x)$ is undecidable. (Proof idea :
            reduce Atm to the problem of deciding number theory).
        \item $Th(N,+)$ is decidable
    \item The set of provable statements in number theory is Turing Recognizable
        
        we can enumerate all the provable statements. (List them all out)
    \item Some statements are true but not provable!
        
    Some statements in  \nth has no proof.
        
    Assume all true statements are provable, Look for a proof of $\phi\ and\ \neg\phi
        $one of then will be true\\
        But \nth is undecidable, hence contradiction!
\end{enumerate}

\textbf{Chapter 7 TIME COMPLEXITY}
\begin{enumerate}[1., leftmargin = 0.6cm]
\itemsep0em
\item  Measuring the complexity
\item let M be a deteministic Turing machine that halts on all inputs. The \imp{running
    time} or \imp{time complexity} of M is the function $f: N \rightarrow N$, where
    $f(n)$ is the runnging time of M, we say that M runs in time $f(n)$ and that M is
    an $f(n)$ time Turing machine. Customarily we use n to represent the lenght of the 
    input.
\item \imp{BIG O and SMALL O}
\item Let $f$ and $g$ be functions $f,g: N \rightarrow R^+$. Say that $f(n) = O(g(n))$
if positive integers c and $n_0$ exists= such that for every integer $n \ge n_0$,\\
    $f(n) \le cg(n)$\\
When $f(n) = O(g(n))$ we say that $g(n)$ is an upper bound for $f(n)$.
\item the big-O interacts wit logarithms in a particular way. $log_bn = log_2n /
log_2b$
\item Let t(n) be a function, where t(n) $\ge$ n then every t(n) time multitape Turing
Machine has an equivalent O($t^2$(n)) time single-tape Turing machine.
\item Let N be a nondeterministic TUring machine that is a decider. The \imp{running
time} of N is the function $f:N \rightarrow N$, where f(n) is the maximum number of
steps that N uses on any branch of its computation of any inputof length n, as shown in
the following figure.

The definition of the running time of a non deterministic TM is not intended to
correspond to any real-world computing device. Rather,it is a useful math definition.

\item Let t(n) be function, where $t(n) \ge n$. Then every t(n) time nondeterministic
signle-tape Turing machine has an equivalent $2^{O(t(n))}$ time deterministic single
tape Turing Machine

\item \imp{the class P} - polynomial.
\item Exponential is bad, polynomial is good.
\item \imp{P} is the class of languages that are decidable in polynomial time on a
deterministic single tape Turing machine. (realistically solvable on a computer).
    \begin{enumerate}[1., leftmargin = 0.6cm]
    \itemsep0em
    \item Every context-free language is a member of P.
    \item \imp{Dynamic programming} - break a problem to smaller subproblems, and solve
    each subproblem only once.
    \item PATH problem is in P
    \end{enumerate}
\item \imp{the class NP} - nondeterministically polynomial, i.e. have P time verifier.
    \begin{enumerate}[1., leftmargin = 0.6cm]
    \itemsep0em
    \item HAMILTONIAN PATH problem is in NP, exponential decider, but Polynomial
    \item Composites ( natural number is composite if it is the product of two integers
    greater than 1, i.e. composite is not a prime number). Can be easily verified with
    given divider.
    \item $\overline{HAMPATH}$ is not in NP
    \end{enumerate}
\item \imp{Verifier} for a language A is an algorithm V, where

\begin{centering}
    $A = \{ w\ |\ V\ accepts \langlew,c\ranlge\ for\ some\ string\ c\}$\\
\end{centering}

We measure the time of a verifier only in terms of the length of $w$, so a
ponynomial time verifier runs in polynomial time in the length of $w$. A language A
is \imp{polynomially verifiable} if it has a polynomial time verifier.
\item Theorem 7.20. A language is in NP iff it is deced by some nondeterministic
polynomial time Turing machine.
    \begin{enumerate}[1., leftmargin = 0.6cm]
    \itemsep0em
    \item Forward convert P time verifier to an equivalent NTM.
    \item Back    convert NTP to P time verifier.
    \end{enumerate}
\item Nondeterministic time complexity class $NTIME(t(n))$
    \begin{enumerate}[1., leftmargin = 0.6cm]
    \itemsep0em
    \item $NTIME(t(n)) = \curl{L\ |\ L\ is\ a\ language\ decided\ by\ an\ O(t(n))\
    time\ nondeterministic\ TM}$
    \end{enumerate}
\item $NP = \bigcup_k NTIME(n^k)$
    \begin{enumerate}[1., leftmargin = 0.6cm]
    \itemsep0em
    \item NP problems
    \item The QLIQUE problem
    \item The SUBSET problem

    \item Proof: provide polynomial time verifier or NTM (nondeterministic turing
    machine).

    \end{enumerate}
\item \imp{P versus NP}
    \begin{enumerate}[1., leftmargin = 0.6cm]
    \itemsep0em
    \item P can be decided quickly.
    \item NP can be verified quickly.
    \item P = NP ? nobody can prove or disprove.
    \item The best deterministic method currently known for deciding languages in NP
    uses exponential time. In other words, we can prove that.\\
        $ NP \subseteq EXPTIME = \bigcup_kTIME(2^{n^k})$
    \end{enumerate}
\item \imp{NP-COMPLETENESS}
    \begin{enumerate}[1., leftmargin = 0.6cm]
    \itemsep0em
    \item If a plynomial time algorithm exists for any of these problems, all problems
    in NP would be polynomial time solvable.
    \item \imp{Satisfiability problem} a boolean formula is satisfiable if some
    assignment of 0s and 1s to the variables makes the formula evaluate to 1.
    \item \imp{POLYNOMIAL TIME REDUCIBILITY}
        \begin{enumerate}[1., leftmargin = 0.6cm]
        \itemsep0em
        \item If $A \le_p B$ and $B\in P$ then $A \in P$
        \end{enumerate}
    \end{enumerate}
    \item A language B is NP-complete if it satisfies two conditions:

        1. B is NP, and

        2. every A in NP is polynomial time reducible to B.if it satisfies two
        conditions:

            1. B is NP, and

            2. every A in NP is polynomial time reducible to B.
    \item if B is NP complete and $B \in P$, then P = NP.
    \item if B is NP complete and $B \le_p C$, then C is NP complete.
    \item \imp{THE COOK-LEVIN THEOREM} the SAT is NP-complete
    \item Other NP complete problems:
        \begin{enumerate}[1., leftmargin = 0.6cm]
        \itemsep0em
        \item CLIQUE
        \item VERTEX-COVER
        \item HAMPATH
        \item UHAMPATH (undirected)
        \item SUBSET SUM PROBLEM

        \end{enumerate}

\item \imp{SPACE COMPLEXITY}
\end{enumerate}
\end{document}
