\documentclass{article}

\usepackage{fancyhdr}
\usepackage{wrapfig}
\usepackage{extramarks}
\usepackage{multicol}
\usepackage{amsmath}
\usepackage{amsthm}
\usepackage{amsfonts}
\usepackage{tikz}
\usepackage{forest,adjustbox}
\usepackage[plain]{algorithm}
\usepackage{algpseudocode}
\usepackage{enumerate}
\usepackage[shortlabels]{enumitem}                     
          \setlist[enumerate, 1]{1\textsuperscript{o}}



\usepackage{listings}
\usepackage{xcolor}
%\lstset { %
%  language=C++,
%  backgroundcolor=\color{black!5}, % set backgroundcolor
%  basicstyle=\footnotesize,% basic font setting
%}

%\usetikzlibrary{automata,positioning}
\usetikzlibrary{positioning,shapes,shadows,arrows,automata} %
% Basic Document Settings
%

\topmargin=-0.45in
\evensidemargin=0in
\oddsidemargin=0in
\textwidth=6.5in
\textheight=9.0in
\headsep=0.25in

\linespread{1.1}

\pagestyle{fancy}
\lhead{\hmwkClass}
\chead{ (\hmwkClassInstructor\ \hmwkClassTime) }
\rhead{\shortName \hspace{0.4cm} \hmwkTitle}
\lfoot{\lastxmark}
\cfoot{\thepage}

\renewcommand\headrulewidth{0.4pt}
\renewcommand\footrulewidth{0.4pt}

\newcommand\curl[1]{\{#1\}}

\setlength\parindent{0pt}

%
% Create Problem Sections
%

\newcommand{\enterProblemHeader}[1]{
  \nobreak\extramarks{}{Problem \arabic{#1} continued on next page\ldots}\nobreak{}
  \nobreak\extramarks{Problem \arabic{#1} (continued)}{Problem \arabic{#1} continued on next page\ldots}\nobreak{}
}

\newcommand{\exitProblemHeader}[1]{
  \nobreak\extramarks{Problem \arabic{#1} (continued)}{Problem \arabic{#1} continued on next page\ldots}\nobreak{}
  \stepcounter{#1}
  \nobreak\extramarks{Problem \arabic{#1}}{}\nobreak{}
}

\setcounter{secnumdepth}{0}
\newcounter{partCounter}
\newcounter{homeworkProblemCounter}
\setcounter{homeworkProblemCounter}{1}
\nobreak\extramarks{Problem \arabic{homeworkProblemCounter}}{}\nobreak{}

%
% Homework Problem Environment
%
% This environment takes an optional argument. When given, it will adjust the
% problem counter. This is useful for when the problems given for your
% assignment aren't sequential. See the last 3 problems of this template for an
% example.
%
\newenvironment{homeworkProblem}[1][-1]{
  \ifnum#1>0
    \setcounter{homeworkProblemCounter}{#1}
  \fi
  \section{Problem \arabic{homeworkProblemCounter}}
  \setcounter{partCounter}{1}
  \enterProblemHeader{homeworkProblemCounter}
}{
  \exitProblemHeader{homeworkProblemCounter}
}

%
% Homework Details
%  - Title
%  - Due date
%  - Class
%  - Section/Time
%  - Instructor
%  - Author
%

\newcommand{\hmwkTitle}{Chapter 3 review}
\newcommand{\hmwkDueDate}{February 22 2016}
\newcommand{\hmwkClass}{CS581 Theory of Computation}
\newcommand{\hmwkClassTime}{Winter 2016}
\newcommand{\hmwkClassInstructor}{Harry H. Porter}
\newcommand{\hmwkAuthorName}{Konstantin Macarenco}
\newcommand{\shortName}{Konstantin M.}

%
% Title Page
%

\title{
  \vspace{2in}
  \textmd{\textbf{\hmwkClass:\ \hmwkTitle}}\\
  \normalsize\vspace{0.1in}\small{Due\ on\ \hmwkDueDate\ at 2:00pm}\\
  \vspace{0.1in}\large{\textit{\hmwkClassInstructor\ \hmwkClassTime}}
  \vspace{3in}
}

\author{\textbf{\hmwkAuthorName}}
\date{}

\renewcommand{\part}[1]{\textbf{\large Part \Alph{partCounter}}\stepcounter{partCounter}\\}
\newcommand{\answ}[1]{\hspace{1cm}\textbf{Answ:} #1}
\newcommand{\problem}[1]{\large{\textbf{Problem #1}}\\}

%
% Various Helper Commands
%

% Useful for algorithms
\newcommand{\alg}[1]{\textsc{\bfseries \footnotesize #1}}

% For derivatives
\newcommand{\deriv}[1]{\frac{\mathrm{d}}{\mathrm{d}x} (#1)}

% For partial derivatives
\newcommand{\pderiv}[2]{\frac{\partial}{\partial #1} (#2)}

% Integral dx
\newcommand{\dx}{\mathrm{d}x}

% Alias for the Solution section header
\newcommand{\solution}{\textbf{\large Solution}}

% Probability commands: Expectation, Variance, Covariance, Bias
\newcommand{\E}{\mathrm{E}}
\newcommand{\Var}{\mathrm{Var}}
\newcommand{\Cov}{\mathrm{Cov}}
\newcommand{\Bias}{\mathrm{Bias}}

\newcommand\Vtextvisiblespace[1][.3em]{%
  \mbox{\kern.06em\vrule height.3ex}%
  \vbox{\hrule width#1}%
  \hbox{\vrule height.3ex}}

\begin{document}

\maketitle

\pagebreak

\begin{enumerate}[1., leftmargin = 0.5cm]
\itemsep0em
    \item Turing recognizable languages - are recognized by some TM.
    \item Turing decidable languages - are decided by some TM.
    \item Turing Machine
    \item Halt and accept - whenever the machine enters the accept state computation immediately halts.
    \item Halt and reject - whenever the machine enters the reject state computation immediately halts.
    \item If an edge is missing assume it leads to reject.
    
        \begin{enumerate}[1., leftmargin = 0.5cm]
        \itemsep0em
        \item $Q$ set of states
        \item $\Sigma$ input alphabet, empty symbol is not a part of $\Sigma$
        \item $\Gamma$ stack alphabet, where $\Sigma \subseteq \Gamma$ and blank $\in \Gamma$
        \item $\delta : Q \times \Gamma \rightarrow Q \times \Gamma \times \curl{R,L}$
        \item $q_{start} \in Q$
        \item $q_{reject} \in Q$
        \item $q_{accept} \in Q$, where $q_{accept} \ne q_{reject}$
        \end{enumerate}
    \item Multitape Turing Machine
        \begin{enumerate}[1., leftmargin = 0.5cm]
        \itemsep0em
        \item $Q$ set of states
        \item $\Sigma^k$ input alphabets, empty symbol is not a part of $\Sigma^k$
        \item $\Gamma^k$ stack alphabets, where $\Sigma^k \subseteq \Gamma^k$ and blank $\in \Gamma^k$
        \item $\delta : Q \times \Gamma^k \rightarrow Q \times \Gamma^k \times \curl{R,L,S}$
        \item $q_{start} \in Q$
        \item $q_{reject} \in Q$
        \item $q_{accept} \in Q$, where $q_{accept} \ne q_{reject}$
        \end{enumerate}

    \item Non-deterministic Turing Machine - accept if at least one branch accepts. Reject if all branches rejects.
        \begin{enumerate}[1., leftmargin = 0.5cm]
        \itemsep0em
        \item $Q$ set of states
        \item $\Sigma$ input alphabets, empty symbol is not a part of $\Sigma$
        \item $\Gamma$ stack alphabets, where $\Sigma \subseteq \Gamma$ and blank $\in \Gamma$
        \item $\delta : Q \times \Gamma \rightarrow P \curl{Q \times \Gamma \times \curl{R,L}}$
        \item $q_{start} \in Q$
        \item $q_{reject} \in Q$
        \item $q_{accept} \in Q$, where $q_{accept} \ne q_{reject}$
        \end{enumerate}

    \item Enumerator \\
        A language is Turing recognizable if some enumerator enumerates it.
        \begin{enumerate}[1., leftmargin = 0.5cm]
        \itemsep0em
        \item $Q$ set of states
        \item $\Sigma$ input alphabets, empty symbol is not a part of $\Sigma$
        \item $\Gamma$ stack alphabets, where $\Sigma \subseteq \Gamma$ and blank $\in \Gamma$
        \item $\delta : Q \times \Gamma \rightarrow Q \times \Gamma \times \curl{R,L} \times \Sigma_{\epsilon}$
        \item $q_{start} \in Q$
        \item $q_{print} \in Q$
        \item $q_{accept} \in Q$, where $q_{accept} \ne q_{print}$
        \end{enumerate}
    
    \item All turing machine modules have equivalent computing power.

    \item Decidable languages: When given an input the TM will always halt. The TM will ACCEPT if it is in L.
        The TM will REJECT if it is not in L.

    \item Turing recognizable language. When given an input, string in the language. The TM will always HALT and ACCEPT.
    OR if the string is not in the lanhuage the TM will either REJECT or LOOP. \\   
        Also recursively enumerable, partially decidable (same as Turing Recognizable).
    
    \item Some languages are not Turing recogniziable (RE, PD)
    
    \item Turing-Church Thesis "Anything Computable == Computable By a TM"

    \item A "Configuration" is a way to represent the entire state of a TM at one moment during computation.

    \item Algorithm definition: Algorithm = Turing Machine \\
        High-Level specification - PSEUDO CODE \\
        Implementation specification - Contents of the tape, data representation, motion of the head. More Detail but still abstract.\\
        TM. Specification - Formal Descritpion, FULLY DETAILED (lowest level)
    
\end{enumerate}

\end{document}
