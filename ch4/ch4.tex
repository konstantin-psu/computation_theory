\documentclass{article}

\usepackage{fancyhdr}
\usepackage{wrapfig}
\usepackage{extramarks}
\usepackage{multicol}
\usepackage{amsmath}
\usepackage{amsthm}
\usepackage{amsfonts}
\usepackage{tikz}
\usepackage{forest,adjustbox}
\usepackage[plain]{algorithm}
\usepackage{algpseudocode}
\usepackage{enumerate}
\usepackage[shortlabels]{enumitem}                     
          \setlist[enumerate, 1]{1\textsuperscript{o}}



\usepackage{listings}
\usepackage{xcolor}
%\lstset { %
%  language=C++,
%  backgroundcolor=\color{black!5}, % set backgroundcolor
%  basicstyle=\footnotesize,% basic font setting
%}

%\usetikzlibrary{automata,positioning}
\usetikzlibrary{positioning,shapes,shadows,arrows,automata} %
% Basic Document Settings
%

\topmargin=-0.45in
\evensidemargin=0in
\oddsidemargin=0in
\textwidth=6.5in
\textheight=9.0in
\headsep=0.25in

\linespread{1.1}

\pagestyle{fancy}
\lhead{\hmwkClass}
\chead{ (\hmwkClassInstructor\ \hmwkClassTime) }
\rhead{\shortName \hspace{0.4cm} \hmwkTitle}
\lfoot{\lastxmark}
\cfoot{\thepage}

\renewcommand\headrulewidth{0.4pt}
\renewcommand\footrulewidth{0.4pt}

\newcommand\curl[1]{\{#1\}}
\newcommand\angl[1]{\langle #1 \rangle}}

\setlength\parindent{0pt}

%
% Create Problem Sections
%

\newcommand{\enterProblemHeader}[1]{
  \nobreak\extramarks{}{Problem \arabic{#1} continued on next page\ldots}\nobreak{}
  \nobreak\extramarks{Problem \arabic{#1} (continued)}{Problem \arabic{#1} continued on next page\ldots}\nobreak{}
}

\newcommand{\exitProblemHeader}[1]{
  \nobreak\extramarks{Problem \arabic{#1} (continued)}{Problem \arabic{#1} continued on next page\ldots}\nobreak{}
  \stepcounter{#1}
  \nobreak\extramarks{Problem \arabic{#1}}{}\nobreak{}
}

\setcounter{secnumdepth}{0}
\newcounter{partCounter}
\newcounter{homeworkProblemCounter}
\setcounter{homeworkProblemCounter}{1}
\nobreak\extramarks{Problem \arabic{homeworkProblemCounter}}{}\nobreak{}

%
% Homework Problem Environment
%
% This environment takes an optional argument. When given, it will adjust the
% problem counter. This is useful for when the problems given for your
% assignment aren't sequential. See the last 3 problems of this template for an
% example.
%
\newenvironment{homeworkProblem}[1][-1]{
  \ifnum#1>0
    \setcounter{homeworkProblemCounter}{#1}
  \fi
  \section{Problem \arabic{homeworkProblemCounter}}
  \setcounter{partCounter}{1}
  \enterProblemHeader{homeworkProblemCounter}
}{
  \exitProblemHeader{homeworkProblemCounter}
}

%
% Homework Details
%  - Title
%  - Due date
%  - Class
%  - Section/Time
%  - Instructor
%  - Author
%

\newcommand{\hmwkTitle}{Chapter 4 review}
\newcommand{\hmwkDueDate}{February 22 2016}
\newcommand{\hmwkClass}{CS581 Theory of Computation}
\newcommand{\hmwkClassTime}{Winter 2016}
\newcommand{\hmwkClassInstructor}{Harry H. Porter}
\newcommand{\hmwkAuthorName}{Konstantin Macarenco}
\newcommand{\shortName}{Konstantin M.}

%
% Title Page
%

\title{
  \vspace{2in}
  \textmd{\textbf{\hmwkClass:\ \hmwkTitle}}\\
  \normalsize\vspace{0.1in}\small{Due\ on\ \hmwkDueDate\ at 2:00pm}\\
  \vspace{0.1in}\large{\textit{\hmwkClassInstructor\ \hmwkClassTime}}
  \vspace{3in}
}

\author{\textbf{\hmwkAuthorName}}
\date{}

\renewcommand{\part}[1]{\textbf{\large Part \Alph{partCounter}}\stepcounter{partCounter}\\}
\newcommand{\answ}[1]{\hspace{1cm}\textbf{Answ:} #1}
\newcommand{\problem}[1]{\large{\textbf{Problem #1}}\\}

%
% Various Helper Commands
%

% Useful for algorithms
\newcommand{\alg}[1]{\textsc{\bfseries \footnotesize #1}}

% For derivatives
\newcommand{\deriv}[1]{\frac{\mathrm{d}}{\mathrm{d}x} (#1)}

% For partial derivatives
\newcommand{\pderiv}[2]{\frac{\partial}{\partial #1} (#2)}

% Integral dx
\newcommand{\dx}{\mathrm{d}x}

% Alias for the Solution section header
\newcommand{\solution}{\textbf{\large Solution}}

% Probability commands: Expectation, Variance, Covariance, Bias
\newcommand{\E}{\mathrm{E}}
\newcommand{\Var}{\mathrm{Var}}
\newcommand{\Cov}{\mathrm{Cov}}
\newcommand{\Bias}{\mathrm{Bias}}

\newcommand\Vtextvisiblespace[1][.3em]{%
  \mbox{\kern.06em\vrule height.3ex}%
  \vbox{\hrule width#1}%
  \hbox{\vrule height.3ex}}

\begin{document}

\maketitle

\pagebreak

\textbf{Decidable languages}
\begin{enumerate}[1., leftmargin = 0.5cm]
\itemsep0em
    \item Building deciders for various languages
    
    \item $A_{DFA} = \curl{\langle B, w \rangle | B \text{ is a DFA that accepts } w }$ - Is decidable. \\
        On input "w" construct B, accept if B accepts, reject if B rejects.

    \item $A_{NFA} = \curl{\langle N, w \rangle | N \text{ is a NFA that accepts } w }$ - Is decidable. \\
        On input "w" convert N into DFA B, run $A_{DFA}$ on $\langle B, w \rangle $ accept if $A_{DFA}$ accepts, otherwise reject.

    \item $A_{REG} = \curl{\langle R, w \rangle | R \text{ is a Regex that generate } w }$ - Is decidable. \\
        On input "w" convert R into NFA N, run $A_{NFA}$ on $\langle N, w \rangle $ accept if $A_{NFA}$ accepts, otherwise reject.

    \item $E_{DFA} \curl{\langle B \rangle | B \text{ DFA and L(B) } = \emptyset}$ \\
        On input $\langle B \rangle$ Recursively mark all the states reachable from the start state. If no accept state is marked - accept, otherwise reject.

    \item $EQ_{DFA}  \curl{\angl{A, B} | A,B \text{ are DFAs and L(A) = L(B)}}$ \\
        On input $\angl{A,B}$ construct symmetric difference C of A and B, and run $E_{DFA}$ on C, accept if $E_{DFA}$ accepts, and reject otherwise.
    
    \item $A_{CFG} = \curl{\langle C, w \rangle | C \text{ is a CFG that generates } w }$ - Is decidable. \\
        On input "w" convert C into chomsky normal form C', generate all strings that can be generated in $2n - 1 $ steps, where
        $n$ is the lenght of $w$, if $w \in Generated$, accept, otherwise
        reject.

    \item $E_{CFG} \curl{\langle C \rangle | C \text{ CFG and L(C) } = \emptyset}$  - is decidable\\
        On input $\angl{C}$. Recursively mark all the strings that can lead to terminals. If start string is marked reject, otherwise accept.

    \item $EQ_{CFG}  \curl{\angl{A, B} | A,B \text{ are CFGs and L(A) = L(B)}}$ - is not decidable\\

    \item Every Question about Regular languages is decidable.

    \item Some questions about CFLs are decidable, but some are not.

    \item The set of all Turing machines is countably infinite.

    \item The set of all Turing-Recognizable languages if countably infinite.

    \item The set of all Languages is uncountably infinite.

\end{enumerate}

\textbf{Undecidable languages}
\begin{enumerate}[1., leftmargin = 0.5cm]
\itemsep0em

    \item  Undecidable examples
        \begin{enumerate}[1., leftmargin = 0.5cm]
        \itemsep0em
        \item Is a CFG ambiguous
        \item Do two CFGs have any strings the can generate in common?
        \item Is the complements of a cfg also a cfg?
        \end{enumerate}
    
    \item $A_{TM} = \curl{\langle M, w \rangle | M \text{ is a TM that accepts } w }$ - Is turing recognizable. (Universal Turing machine)\\
        On input "w" construct and run M, accept if M accepts, reject or halt if M rejects or halts.

    \item Countability proof by Cantors diagonalization method.
    \item Onto (every element in A has corresponding element in B) vs one-to-one(every element in A corresponds to a unique element in B) function. 
    \item Correspondence - onto and one-to-one.
    \item Infinite number of languages is infinite and not countable, number of TM is infinitely countable therefore some languages are not Turing
        Recongnizable

    \item Halting problem - proof that $A_{TM}$ is not decidable.
        \begin{enumerate}[a., leftmargin = 0.5cm]
        \itemsep0em
            \item Assume that $A_{TM}$ is decidable.
            \item Construct TM H that decides $A_{TM}$\\
                H on input $\angl{M,w}$ accepts if M accepts, and rejects if M rejects.
                
            \item Construct a TM D that takes a description of a turing machine as an input $\angl{M}$, and uses H as subroutine in the following way: \\
                It runs H on inputs $\angl{M,M}$, and accepts if H rejects, and rejects if H accepts.
            \item  Run D on description of itself the we get a contradiction:\\
                D accepts, when D rejects D, and \\
                D rejects, when D accepts D. \\
                \textbf{PARADOX}
            \item Hence we get a contradiction, therefore $A_{TM}$ is not Decidable.
        \end{enumerate}

    \item If complement of a language is recognizable then Language is called Turing Co-recognizable.

    \item If a languages is turing recognizable and turing co-recognizable, then this language is decidable.
\end{enumerate}

\end{document}
