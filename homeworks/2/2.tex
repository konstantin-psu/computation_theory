\documentclass{article}

\usepackage{fancyhdr}
\usepackage{wrapfig}
\usepackage{extramarks}
\usepackage{multicol}
\usepackage{amsmath}
\usepackage{amsthm}
\usepackage{amsfonts}
\usepackage{tikz}
\usepackage[plain]{algorithm}
\usepackage{algpseudocode}
\usepackage{enumerate}
\usepackage[shortlabels]{enumitem}                                          
                    \setlist[enumerate, 1]{1\textsuperscript{o}}



\usepackage{listings}
\usepackage{xcolor}
%\lstset { %
%    language=C++,
%    backgroundcolor=\color{black!5}, % set backgroundcolor
%    basicstyle=\footnotesize,% basic font setting
%}
\lstset{
  basicstyle=\itshape,
  xleftmargin=3em,
  literate={->}{$\rightarrow$}{2}
           {α}{$\alpha$}{1}
           {δ}{$\delta$}{1}
}

%\usetikzlibrary{automata,positioning}
\usetikzlibrary{positioning,shapes,shadows,arrows,automata}

%
% Basic Document Settings
%

\topmargin=-0.45in
\evensidemargin=0in
\oddsidemargin=0in
\textwidth=6.5in
\textheight=9.0in
\headsep=0.25in

\linespread{1.1}

\pagestyle{fancy}
\lhead{\hmwkClass}
\chead{ (\hmwkClassInstructor\ \hmwkClassTime) }
\rhead{\shortName \hspace{0.4cm} \hmwkTitle}
\lfoot{\lastxmark}
\cfoot{\thepage}

\renewcommand\headrulewidth{0.4pt}
\renewcommand\footrulewidth{0.4pt}

\setlength\parindent{0pt}

%
% Create Problem Sections
%

\newcommand{\enterProblemHeader}[1]{
    \nobreak\extramarks{}{Problem \arabic{#1} continued on next page\ldots}\nobreak{}
    \nobreak\extramarks{Problem \arabic{#1} (continued)}{Problem \arabic{#1} continued on next page\ldots}\nobreak{}
}

\newcommand{\exitProblemHeader}[1]{
    \nobreak\extramarks{Problem \arabic{#1} (continued)}{Problem \arabic{#1} continued on next page\ldots}\nobreak{}
    \stepcounter{#1}
    \nobreak\extramarks{Problem \arabic{#1}}{}\nobreak{}
}

\setcounter{secnumdepth}{0}
\newcounter{partCounter}
\newcounter{homeworkProblemCounter}
\setcounter{homeworkProblemCounter}{1}
\nobreak\extramarks{Problem \arabic{homeworkProblemCounter}}{}\nobreak{}

%
% Homework Problem Environment
%
% This environment takes an optional argument. When given, it will adjust the
% problem counter. This is useful for when the problems given for your
% assignment aren't sequential. See the last 3 problems of this template for an
% example.
%
\newenvironment{homeworkProblem}[1][-1]{
    \ifnum#1>0
        \setcounter{homeworkProblemCounter}{#1}
    \fi
    \section{Problem \arabic{homeworkProblemCounter}}
    \setcounter{partCounter}{1}
    \enterProblemHeader{homeworkProblemCounter}
}{
    \exitProblemHeader{homeworkProblemCounter}
}

%
% Homework Details
%   - Title
%   - Due date
%   - Class
%   - Section/Time
%   - Instructor
%   - Author
%

\newcommand{\hmwkTitle}{Homework\ \#2}
\newcommand{\hmwkDueDate}{February 1 2016}
\newcommand{\hmwkClass}{CS581 Theory of Computation}
\newcommand{\hmwkClassTime}{Winter 2016}
\newcommand{\hmwkClassInstructor}{Harry H. Porter}
\newcommand{\hmwkAuthorName}{Konstantin Macarenco}
\newcommand{\shortName}{Konstantin M.}

%
% Title Page
%

\title{
    \vspace{2in}
    \textmd{\textbf{\hmwkClass:\ \hmwkTitle}}\\
    \normalsize\vspace{0.1in}\small{Due\ on\ \hmwkDueDate\ at 2:00pm}\\
    \vspace{0.1in}\large{\textit{\hmwkClassInstructor\ \hmwkClassTime}}
    \vspace{3in}
}

\author{\textbf{\hmwkAuthorName}}
\date{}

\renewcommand{\part}[1]{\textbf{\large Part \Alph{partCounter}}\stepcounter{partCounter}\\}
\newcommand{\answ}[1]{\hspace{1cm}\textbf{Answ:} #1}
\newcommand{\problem}[1]{\large{\textbf{Problem #1} \\}}

%
% Various Helper Commands
%

% Useful for algorithms
\newcommand{\alg}[1]{\textsc{\bfseries \footnotesize #1}}

% For derivatives
\newcommand{\deriv}[1]{\frac{\mathrm{d}}{\mathrm{d}x} (#1)}

% For partial derivatives
\newcommand{\pderiv}[2]{\frac{\partial}{\partial #1} (#2)}

% Integral dx
\newcommand{\dx}{\mathrm{d}x}

% Alias for the Solution section header
\newcommand{\solution}{\textbf{\large Solution}}

% Probability commands: Expectation, Variance, Covariance, Bias
\newcommand{\E}{\mathrm{E}}
\newcommand{\Var}{\mathrm{Var}}
\newcommand{\Cov}{\mathrm{Cov}}
\newcommand{\Bias}{\mathrm{Bias}}

\begin{document}

\maketitle

\pagebreak

\problem{2.6}
Give context-free grammars generating the following languages. \\  \\
\problem{2.6 b}
The complement of the language $\{a^n b^n | n \geq 0\}$ \\ \\

\begin{table}[h!]
\centering
\begin{tabular}{l}
$S \rightarrow a S b\: | \:b Y \:|\: Y a $\\
$Y \rightarrow b Y \:| \:a Y \:| \:\epsilon$
\end{tabular}
\end{table}

\problem{2.6 d}
$\{x_1 \# x_2 \# \cdots \# x_k |  k \geq 1, \text{ each }x_i \in \{a,b\}^*,\text{ and for some } i \text{ and } j, x_i 
= {x_j}^R\}$

\begin{table}[h!]
\centering
\begin{tabular}{l}
$S \rightarrow A B C $\\
$A \rightarrow D \# A \:| \: \epsilon $\\
$B \rightarrow 0 B 0 \: | \: 1 B 1 \: | \: E $\\
$C \rightarrow \#DC \:| \: \epsilon $\\
$D \rightarrow 1D \:| \: 0D \: | \: \epsilon $\\
$E \rightarrow \#DE \:| \: \# $\\
\end{tabular}
\end{table}



\problem{2.7}
Give informal English description of PDAs for the languages in Exercise 2.6 \\ \\
\problem{2.7 b}
The complement of the language $\{a^n b^n | n \geq 0\}$ \\ \\
\problem{2.7 d}
$\{x_1 \# x_2 \# \cdots \# x_k |  k \geq 1, \text{ each }x_i \in \{a,b\}^*,\text{ and for some } i \text{ and } j, x_i 
= {x_j}^R\}$ \\ \\

\problem{2.9}
Give a context-free grammar that generates the language
\begin{center}
$A = \{a^i b^j c^k | i = j \text{ or } j = k \text{ where } i,j,k \geq 0\}$
\end{center}
Is your grammar ambiguous? Why or why not?

\begin{table}[h!]
\centering
\begin{tabular}{l}
$S \rightarrow A B\: | \: C D $\\
$A \rightarrow aAb \: | \: \epsilon $\\
$B \rightarrow cB \: | \: \epsilon $\\
$C \rightarrow aC \: | \: \epsilon $\\
$D \rightarrow bDc \:| \: \epsilon $\\
\end{tabular}
\end{table}

The language is not ambiguous... \\

\pagebreak

\problem{2.13}
Let $G = (V, \sum, R, S)$ be the following grammar. $V = \{S<T U\}; \sum = \{0,\#\};$ and R is the set of rules:
\begin{table}[h!]
\centering
\begin{tabular}{l}
S $\rightarrow$ $TT$ $|$ $U$ \\
T $\rightarrow$ $0T$ $|$ $T0$ $|$ $\#$   \\
U $\rightarrow$  $0U00$ $|\#$
\end{tabular}
\end{table}

\vspace{0.2cm}

\problem{2.13 a}
Describe $L(G)$ in English. \\

Informally $L(G)$ is either two or more $\#$ separated by arbitrary number of 0's (zero or more)
or zero or more zero followed by $\#$ and by twice as many zeros as in before $\#$. 
More formally it is $\{0^{i_1}_1\#0^{i_2}_2\#0^{i_3}_3\#0^{i_4}_4\#\cdots\#0^{i_n}_k\ |
\text{ where } i_j \geq 0 \text{ and } k \geq 3 \}$ or $\{0^n \# 0^{2n} | n \geq 1\}$ \\ \\
\problem{2.13 b}
Prove that $L(G)$ is not regular. \\ \\

\problem{2.15}
Give a counterexample to show that the following construction fails to prove that the class of context-free languages is closed under star.
Let A be a CFL that is generated by th CFG $G = (V,\sum,R,S)$. Add the new rule $S \rightarrow SS$ and call the
resulting grammar $G'$A. This grammar is supposed to generate $A*$. \\ \\

\problem{2.19}
Let  CFG G be the following grammar.
\begin{table}[h!]
\centering
\begin{tabular}{l}
$S \rightarrow a S b\: | \:b Y \:|\: Y a $\\
$Y \rightarrow b Y \:| \:a Y \:| \:\epsilon$
\end{tabular}
\end{table}

Give a simple description of L(G) in English. Use that description to give a CFG for $\overline{L(G)}$, the compliment of $L(G)$. \\

L(G) is the language that produces all strings not in $a^nb^n$, i.e. compliment of of $a^nb^n$.

\begin{table}[h!]
\centering
\begin{tabular}{l}
$S \rightarrow a S b\: | \: \epsilon$
\end{tabular}
\end{table}

\problem{2.28}

Give unambiguous CFGs for the following languages.\\ \\
\problem{2.28 a}
$\{w| \text{ in every prefix of } w \text{ the number of a's is at least the number of b's }\}$ \\ \\
\problem{2.28 b}
$\{w| \text{ the number of a's and the number of b's in } w \text{ are equal }\}$ \\ \\
\problem{2.28 c}
$\{w| \text{ the number of a's is at least the number of b's in } w\}$ \\ \\


\problem{2.30}
Use the pumping lemma to show that the following languages are not context free. \\ \\
\problem{2.30 a}
$\{ 0^n 1^n 0^n 1^n | n \geq 0  \}$ \\ \\
\problem{2.30 d}
$\{ t_1 \# t_2 \# \cdots \# t_k | k \geq 2,\text{ each } t_i \in \{a,b\}^*,\text{ and }t_i = t_j\text{ for some } i \neq j  \}$ \\ \\
\problem{2.31}
Let B be the language of all palindromes over $\{0,1\}$ containing equal numbers of 0s and 1s. Show that B is not context free. \\ \\
\problem{2.33}
Show that $F = \{a^i, b^j | i = kj\text{ for some positive integer } k\}$ is not context free. \\ \\
\problem{2.35}
Let G be a CFG in Chomsky normal form that contains $b$ variables. Show that if G generates some string with a derivation having 
at least $2^b$ steps, $L(G)$ is infinite. \\ \\
\problem{2.46}
Consider the following CFG G:
\begin{table}[h!]
\centering
\begin{tabular}{l}
$S \rightarrow SS \:| \:T$ \\
$T \rightarrow aTb \:| \:ab$
\end{tabular}
\end{table}
Describe $L(G)$ and show that G is ambiguous. Give an unambiguous grammar H where $L(H) = L(G)$ and
sketch a proof that $H$ is unambiguous.


%\begin{homeworkProblem}
%\end{homeworkProblem}
\end{document}
