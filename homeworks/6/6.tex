\documentclass{article}

\usepackage{fancyhdr}
\usepackage{wrapfig}
\usepackage{extramarks}
\usepackage{multicol}
\usepackage{amsmath}
\usepackage{amsthm}
\usepackage{amsfonts}
\usepackage{tikz}
\usepackage{forest,adjustbox}
\usepackage[plain]{algorithm}
\usepackage{algpseudocode}
\usepackage{enumerate}
\usepackage[shortlabels]{enumitem}                     
          \setlist[enumerate, 1]{1\textsuperscript{o}}



\usepackage{listings}
\usepackage{xcolor}
\lstset { %
    language=python,
    frame=single,
    showstringspaces=false,
    backgroundcolor=\color{black!5}, % set backgroundcolor
    basicstyle=\footnotesize,% basic font setting
}

%\usetikzlibrary{automata,positioning}
\usetikzlibrary{positioning,shapes,shadows,arrows,automata} %
% Basic Document Settings
%

\topmargin=-0.45in
\evensidemargin=0in
\oddsidemargin=0in
\textwidth=6.5in
\textheight=9.0in
\headsep=0.25in

\linespread{1.1}

\pagestyle{fancy}
\lhead{\hmwkClass}
\chead{ (\hmwkClassInstructor\ \hmwkClassTime) }
\rhead{\shortName \hspace{0.4cm} \hmwkTitle}
\lfoot{\lastxmark}
\cfoot{\thepage}

\renewcommand\headrulewidth{0.4pt}
\renewcommand\footrulewidth{0.4pt}

\newcommand\curl[1]{\{#1\}}

\setlength\parindent{0pt}

%
% Create Problem Sections
%

\newcommand{\enterProblemHeader}[1]{
  \nobreak\extramarks{}{Problem \arabic{#1} continued on next page\ldots}\nobreak{}
  \nobreak\extramarks{Problem \arabic{#1} (continued)}{Problem \arabic{#1} continued on next page\ldots}\nobreak{}
}

\newcommand{\exitProblemHeader}[1]{
  \nobreak\extramarks{Problem \arabic{#1} (continued)}{Problem \arabic{#1} continued on next page\ldots}\nobreak{}
  \stepcounter{#1}
  \nobreak\extramarks{Problem \arabic{#1}}{}\nobreak{}
}

\setcounter{secnumdepth}{0}
\newcounter{partCounter}
\newcounter{homeworkProblemCounter}
\setcounter{homeworkProblemCounter}{1}
\nobreak\extramarks{Problem \arabic{homeworkProblemCounter}}{}\nobreak{}

%
% Homework Problem Environment
%
% This environment takes an optional argument. When given, it will adjust the
% problem counter. This is useful for when the problems given for your
% assignment aren't sequential. See the last 3 problems of this template for an
% example.
%
\newenvironment{homeworkProblem}[1][-1]{
  \ifnum#1>0
    \setcounter{homeworkProblemCounter}{#1}
  \fi
  \section{Problem \arabic{homeworkProblemCounter}}
  \setcounter{partCounter}{1}
  \enterProblemHeader{homeworkProblemCounter}
}{
  \exitProblemHeader{homeworkProblemCounter}
}

%
% Homework Details
%  - Title
%  - Due date
%  - Class
%  - Section/Time
%  - Instructor
%  - Author
%

\newcommand{\hmwkTitle}{Homework\ \#6}
\newcommand{\hmwkDueDate}{March 16 2016}
\newcommand{\hmwkClass}{CS581 Theory of Computation}
\newcommand{\hmwkClassTime}{Winter 2016}
\newcommand{\hmwkClassInstructor}{Harry H. Porter}
\newcommand{\hmwkAuthorName}{Konstantin Macarenco}
\newcommand{\shortName}{Konstantin M.}

%
% Title Page
%

\title{
  \vspace{2in}
  \textmd{\textbf{\hmwkClass:\ \hmwkTitle}}\\
  \normalsize\vspace{0.1in}\small{Due\ on\ \hmwkDueDate\ at 12:30pm}\\
  \vspace{0.1in}\large{\textit{\hmwkClassInstructor\ \hmwkClassTime}}
  \vspace{3in}
}

\author{\textbf{\hmwkAuthorName}}
\date{}

\renewcommand{\part}[1]{\textbf{\large Part \Alph{partCounter}}\stepcounter{partCounter}\\}
\newcommand{\answ}[1]{\hspace{1cm}\textbf{Answ:} #1}

\newcommand{\domino}[2]{\left [ \cfrac{#1}{#2} \right ]}
\newcommand{\tile}[2]{\cfrac{#1}{#2}}
\newcommand{\problem}[1]{\large{\textbf{Problem #1}}\\}

%
% Various Helper Commands
%

% Useful for algorithms
\newcommand{\alg}[1]{\textsc{\bfseries \footnotesize #1}}

% For derivatives
\newcommand{\deriv}[1]{\frac{\mathrm{d}}{\mathrm{d}x} (#1)}

% For partial derivatives
\newcommand{\pderiv}[2]{\frac{\partial}{\partial #1} (#2)}

% Integral dx
\newcommand{\dx}{\mathrm{d}x}

% Alias for the Solution section header
\newcommand{\solution}{\textbf{\large Solution}}

% Probability commands: Expectation, Variance, Covariance, Bias
\newcommand{\E}{\mathrm{E}}
\newcommand{\Var}{\mathrm{Var}}
\newcommand{\Cov}{\mathrm{Cov}}
\newcommand{\Bias}{\mathrm{Bias}}

\newcommand\Vtextvisiblespace[1][.3em]{%
  \mbox{\kern.06em\vrule height.3ex}%
  \vbox{\hrule width#1}%
  \hbox{\vrule height.3ex}}

\begin{document}

\maketitle

\pagebreak

\problem{6.1}

Give an example in the spirit of the recursion theorem of a program in a real programming 
language (or a reasonable approximation thereof) that prints itself out.\\

\textbf{Solution in python:}

\begin{lstlisting}
x = r"%sprint ('x = r\"' + x) %% (x + '\"\n')"
print ('x = r\"' + x) % (x + '\"\n')
\end{lstlisting}

\vspace{2cm}

\problem{6.11}

Let  $\phi_{eq}$ be defined as in Problem 6.10. Give a model of the sentence
\begin{align}
    \phi_{lt} &= \phi_{eq} \\
    &\wedge \forall x,y [R_1(x,y) \rightarrow \neg R_2(x,y)] \\
    &\wedge \forall x,y [\neg R_1(x,y) \rightarrow (R_2(x,y) \oplus R_2(x,z))] \\
    &\wedge \forall x,y,z [(R_2(x,y) \wedge R_2(y,z)) \rightarrow R_2(x,z)]\\
    &\wedge \forall x \exists y [$_2(x,y)].
\end{align}

\textbf{Solution}\\

One model is $(N,R_1,R_2,\oplus)$, where $R_1$ is equality, $R_2$ is $ < $ and $\oplus $ is $\vee$.
% Solution for 6.10
%The statement $\phi_{eq}$ gives the three conditions of an equivalence relation. A model
%(A,R_1), where A is any universe and R_1 is any equivalence relation over A, is a model of 
%$\phi_{eq}$. For example, let A be the integers $Z$ and let $R_1 = \curl{(i,i) | i \in Z}$.

\vspace{2cm}

\problem{7.1}
Answer each part TRUE or FALSE.

    \begin{table}[h!]
    \centering
    \begin{tabular}{lll}
       a. & $2n = O(n)$             & \textbf{TRUE}  \\
       b. & $n^2 = O(n)$            & \textbf{FALSE} \\
       c. & $n^2 = O(n \  log^2n) $ & \textbf{FALSE} \\
       d. & $n\ log n = O(n^2)$     & \textbf{TRUE}  \\
       e. & $3^n = 2^{O(n)}$        & \textbf{TRUE}  \\
       f. & $2^{2^n} = O(2^{2^n})$  & \textbf{TRUE} 
    \end{tabular}
    \end{table}

\pagebreak

\problem{7.4}
Fill out the table described in the polynomial time algorithm for context-free language
recognition from theorem 7.16 for string $w = baba$ and CFG G:\\
    \begin{table}[h!]
    \centering
    \begin{tabular}{lll}
     $S$     & $\rightarrow$ &  $RT    $ \\
     $R$     & $\rightarrow$ &  $TR | a$ \\
     $T$     & $\rightarrow$ &  $TR | b$
    \end{tabular}
    \end{table}

\textbf{Solution}\\

    \begin{table}[h!]
    \centering
    \begin{tabular}{|l|l|l|l|l|} \hline
         & 1 & 2   & 3 & 4 \\ \hline
       1 & T & T,R & S & S,R,T \\ \hline
       2 &   & R   & S & S \\ \hline
       3 &   &     & T & T,R \\ \hline
       4 &   &     &   & R \\ \hline
    \end{tabular}
    \end{table}

    Because table(1,4) contains S, the TM accepts $w$. \\

\vspace{2cm}
\problem{7.5}
    Is the following formula satisfiable? \\ 

    \begin{centering}
    $(x \vee y) \wedge (x \vee \overline{y}) \wedge (\overline{x} \vee y) \wedge (\overline{x} \vee \overline{y})$ \\
    \end{centering}


\textbf{Solution}\\

    \begin{table}[h!]
    \centering
    \begin{tabular}{cc|c|c|c|c|c}
    x & y & $(x \vee y)$ & $(x \vee \overline{y})$  & $(\overline{x} \vee y)$ & $(\overline{x} \vee \overline{y})$ & Result (conjunction of all) \\ \hline
    T & T &     T        &    T                     &    T                    &  F      &  F\\
    T & F &     T        &    T                     &    F                    &  T      &  F\\
    F & T &     T        &    F                     &    T                    &  T      &  F\\
    F & F &     F        &    T                     &    T                    &  T      &  F\\
    \end{tabular}
    \end{table}

    Hence the formula is not satisfiable.

\pagebreak

\problem{7.12}
Call graphs G and H \textbf{isomorphic} if the nodes of G may be reordered so that it is identical to H. 
Let $ISO = \curl{\langle G, H \rangle | G \text{ and } H \text{ are isomorphic graphs}}$. Show that
$ISO \in NP$ \\

\textbf{Solution}\\

A nondeterministic polynomial time algorithm for $ISO$ operates as follows:\\
`` On input $\langle G, H \rangle$ where $G$ and $H$ are undirected graphs:
    \begin{enumerate}[1., leftmargin = 0.6cm]
    \itemsep0em
    \item  Let $m$ be the number of nodes of $G$ and $H$. If they don't have the same number of nodes, $reject$.
    \item Nondeterministically select a permutation $\pi$ of $m$ elements.
    \item For each pair of nodes $x$ and $y$ of $G$ check that (x,y) is an edge of $G$ iff $(\pi(x), \pi(y))$
        is an edge of $H$. If all agree, $accept$. If any differ, $reject$.''
    \end{enumerate}

    Stage 2 can be implemented in polynomial time nondeterministically, Stage 3 takes polynomial time.
    Therefore $ISO \in NP$
 
\end{document}
